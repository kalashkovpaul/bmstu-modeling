 Целью данной работы является разработка программы с графическим интерфейсом для определения среднего относительного времени пребывания системы в предельном стационарном состоянии, а также вероятности нахоождения системы в указанных состояниях и время до стабилизации вероятностей состояний. Интенсивности переходов из состояния в состояние задаютсся в виде матрицы (предусмотреть определение размера матрицы).

\section*{Марковский процесс}
	
Случайный процесс называется марковским случайным прооцессом, если для каждомго момента времени вероятность люого состояния системы в будущем зависит только от состояния системы в настоящем и не завписит от того, когда и каким образом система пришла в это состояние.

Для марковского случайного процесса составляют уравнения Колмогорова, следуя следующему правилу: в левой части каждого уравнения находится производная функции, отражающей вероятность нахождения системы в $i$-ом состоянии, в правой части находится столько членов, сколько трелок связано с данным состоянием в направленном графе состояний, причём если стрелка выходит из состояния, член имеет знак минус, если в состояние, знак плюс. 

Таким образом, в правой части находится сумма произведений вероятностей всех состояний, переводящих систему в данное состояние, на интенсивности соответствующих переходов, минус суммарная интенсивность всех переходов, выводящих систему из данного состояния, умноженная на вероятность данного состояния. Уравнение Колмогорова для состояния с номером $i$ будет иметь следующий вид:

\begin{equation}
	p^{'}_{i}(t) = \sum_{j=1}^{n}\lambda_{ji}p_{j}(t) - p_{i}(t)\cdot \sum_{j=1}^{n}\lambda_{ij},
\end{equation}

где:

 $n$ --- число состояний рассматриваемой ситемы;

$\lambda_{ij}$ --- интенсивность перехода системы из $i$-го состояния в $j$-ое.


\subsection*{Определение предельных вероятностей}
Предельная вероятность состояния --- среднее относительное время нахождения системы в данном состоянии.
Для определения предельных вероятностей необходимо решить систему уравнений Колмогорова. 
Поскольку по условию задачи рассматриваемый марковский процесс является стационарным, производные вероятностей заменяются нулями. 
При этом одно из уравнений в системе необходимо заменить уравнением нормировки: $\sum_{i = 1}^{n}p_{i}(t) = 1$, где $n$ --- количество состояний системы.

\subsection*{Время стабилизации вероятностей состояний}

Время стабилизации вероятности нахождения системы в состоянии определяется следующим образом: определяются верятности нахождения системы в состоянии $i$ в момент времени $t$ и в момент времени $t + \varDelta t$ с увеличением значения t до тех пор, пока не будет выполняться следующее условие:

\begin{equation*}
 | p_{i}(t) - p_{i}(t + \varDelta t))| < \varepsilon,
\end{equation*}

где $\varepsilon$ --- заданная точность.

\section*{Результаты работы}

\subsection*{Детали реализации}

В листинге \ref{lst:kolmogorov} представлена реализация функции получения коэффициентов уравнений Колмогорова, в листинге \ref{lst:probs} реализация функции получения предельных верятностей, а в листингах \ref{lst:stab_time-1} и \ref{lst:stab_time-2} реализация функции получения среднего относительного времени.
\clearpage

\begin{center}
\captionsetup{justification=raggedright,singlelinecheck=off}
\begin{lstlisting}[label=lst:kolmogorov,caption=Реализация функции получения коэффициентов уравнений Колмогорова]
def __get_kolmogorov_koeffs(self):
    factors = []

    for state in range(self.number_states):
        factors.append([0] * self.number_states)
        for i in range(self.number_states):
            if i != state:
                factors[state][i] = self.intensity_matrix[i][state]
                factors[state][state] -= self.intensity_matrix[state][i]

    return factors
\end{lstlisting}
\end{center}

\begin{center}
\captionsetup{justification=raggedright,singlelinecheck=off}
\begin{lstlisting}[label=lst:probs,caption=Реализация функции получения предельных вероятностей]
def get_limit_probabilities(self):
    coeffs = self.__get_kolmogorov_koeffs()
    coeffs[0] = [1] * self.number_states
    free_numbers = [0] * self.number_states
    free_numbers[0] = 1

    return linalg.solve(coeffs, free_numbers)
\end{lstlisting}
\end{center}

\begin{center}
\captionsetup{justification=raggedright,singlelinecheck=off}
\begin{lstlisting}[label=lst:stab_time-1,caption=Реализация функции получения среднего относительного времени (часть 1)]
def get_time_to_stable(self, probs):
    time = arange(0, max_time, time_delta)

    start_probabilities = [0] * self.number_states
    start_probabilities[0] = 1

    coeffs = self.__get_kolmogorov_koeffs()

    integrated_probabilities = transpose(odeint(self.__get_derivatives,
            start_probabilities,
            time, args=(coeffs)))
\end{lstlisting}
\end{center}

\begin{center}
\captionsetup{justification=raggedright,singlelinecheck=off}
\begin{lstlisting}[label=lst:stab_time-2,caption=Реализация функции получения среднего относительного времени (часть 2)]
	stabilization_time = []

    for state in range(self.number_states):
        probabilities = integrated_probabilities[state]

        for i, probability in enumerate(probabilities):
            if abs(probs[state] - probability) < eps:
                stabilization_time.append(time[i])
                break

            if i == len(probabilities) - 1:
                stabilization_time.append(0)

    return stabilization_time
\end{lstlisting}
\end{center}

\subsection*{Примеры работы}

На рисунках \ref{img:example-1} и \ref{img:example-2} представлены примеры работы разработанной программы для систем из трёх и пяти состояний.

\imgHeight{90mm}{example-1}{Пример работы программы для системы из трёх состояний}

\imgHeight{90mm}{example-2}{Пример работы программы для системы из пяти состояний}
\section*{Вывод}

В ходе выполнения лабораторной работы была реализована программма с графическим интерфейсом для определения среднего относительного времени пребывания системы в предельном стационарном состоянии, а также вероятности нахоождения системы в указанных состояниях.

