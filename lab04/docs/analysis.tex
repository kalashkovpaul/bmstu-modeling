 Целью данной работы является разработка программы с графическим интерфейсом для моделирования системы массового обслуживания (СМО) при помощи принципа $\Delta t$ и событийного принципа и определения максимальной длины очереди, при которой не будет потери сообщений. Рассматриваемая СМО состоит из генератора сообщений, очереди ожидающих обработки сообщений и обслуживающего аппарата (ОА). Генерация сообщений происходит по равномерному закону распределения, время обработки сообщений --- согласно закону распределения Эрланга. Необходимо предоставить возможности ручного задания необходимых параметров, а также возможности возврата обработанного сообщения в очередь обработки с заданной вероятностью.
 
\section*{Принципы работы управляющей программы модели}

\subsection*{Принцип $\Delta t$}

Принцип $\Delta t$ заключается в последовательном анализе состояний всех блоков системы в момент времени $t + \Delta t$ по заданному состоянию блоков $\Delta t$.
 При этом новое состояние определеяется в соответствии с их алгоритмическим описанием с учётом случайных факторов, задаваемых распределениями вероятностей. 
В результате анализа принимается решение о том, какие общесистемные события должны имитироваться в программной модели на данный момент времени.

Данный принцип имеет недостаток: значительные затраты машинного времени на анализ и контрооль всей системы.
Также при недостаточно малом  $\Delta t$ появляется опасность пропуска отдельных событий в системе, что исключает возмоожность молучения правильных результатов при моделировании.

\subsection*{Событийный принцип}

Характерным свойством, отличающим событийный принцип, является изменение состояний свойств в дискретные моменты времени, совпадающие с моментами поступления сигналов окончания или аварийных сигналов. 
Состояния всех блоков программной (имитационной) модели анализируются лишь в момент появления события.
Момент наступления нового события определяется минимальным значением из списка будущих событий, представляющих собой совокупность ближайшего изменения состояния каждого блока системы.

\section*{Используемые законы распределения}

\subsection*{Закон появления сообщений}

Согласно заданию лабораторной работы для генерации сообщений используется равномерный закон распределения.
Случайная величина имеет равномерное распределение на отрезке $[a, b]$, если её функция плотности $p(x)$ имеет вид:

\begin{equation}
	\label{for:equal-1}
    p(x) = 
    \begin{cases}
        \frac{1}{b - a}, \text{если } x \in [a, b],\\
        0, \text{иначе.} \\
    \end{cases}
\end{equation}

Функция распределения $F(x)$ равномерной случайной величины имеет вид:

\begin{equation}
	\label{for:equal-2}
    F(x) = 
    \begin{cases}
    	0, \text{если } x  < a, \\
        \frac{x - a}{b - a}, \text{если } a < x  b,\\
        1, \text{если } x > b. 
    \end{cases}
\end{equation}

Интервал времени между появлением $i$-ого и $(i - 1)$-ого сообщения по равномерному закону распределения вычисляется следующим образом:

\begin{equation}
	T_{i} = a + (b - a) \cdot R,
\end{equation}

\noindentгде $R$ --- псевдослучайное число от 0 до 1.

\section*{Закон обработки сообщений}

Для моделирования работы генератора сообщений в лабораторной работе используется распределение Эрланга. 
Случайная величина имеет распределение Эрланга, если её функция плотности $p(x)$ имеет вид:

\begin{equation}
p(x) = \frac{\lambda^k x^{k-1} e^{-\lambda x} } {(k-1)!}
\end{equation}

В распределении Эрланга целочисленный положительный параметр $k$ 	--- параметр формы (т. е. он влияет на форму распределения,а не просто сдвигает его, как параметр местоположения, или растягивает его или сжимает, как параметр масштаба), а параметр $\lambda$ --- параметр скорости (т. е. он обратен параметру масштаба, отвечающему за растягивание или сжатие графика распределения). 

Функция распределения $F(x)$ нормальной случайной величины имеет вид:

\begin{equation*}
F(x) = 1 - \sum_{i=0}^k  \frac{1}{i!} e^{-\lambda x} (\lambda x)^n
\end{equation*}

Интервал времени между появлением $i$-ого и $(i - 1)$-ого сообщения по равномерному закону распределения вычисляется следующим образом:

\begin{equation}
	T_{i} = - \frac{1}{k \lambda} \sum_{j = 1}^{k} ln (1 - R_i),
\end{equation}

\noindentгде $R_j$ --- псевдослучайное число от 0 до 1.

\section*{Результаты работы}

\subsection*{Детали реализации}

В листинге \ref{lst:delta-t} представлена реализация управляющей программы принципа $\Delta t$, а в листинге \ref{lst:event-driven} --- реализация управляющей программы событийного принципа.
Реализации функций вычисления интервала времени между появлениями сообщений по равномерному закону распределения и времени обработки сообщения поо закону распределения Эрланга представлены в листингах \ref{lst:normal} и \ref{lst:erlang} соотвественно.
\clearpage

\begin{center}
\captionsetup{justification=raggedright,singlelinecheck=off}
\begin{lstlisting}[label=lst:delta-t,caption=Реализация управляющей программы принципа $\Delta t$]
def delta_t(self):
    max_length = 0
    queue_length = 0
    processed_amount = 0
    self.handler.free = True
    handling_time = 0
    current_time = self.step
    generated_time = self.generator.get_time_interval()
    previous_generated_time = 0
    while processed_amount < self.messages_amount:
        if current_time > generated_time:
            queue_length += 1
            if queue_length > max_length:
                max_length = queue_length
            previous_generated_time = generated_time
            generated_time += self.generator.get_time_interval()
        if current_time > handling_time:

            if queue_length > 0:
                handler_was_free = self.handler.free
                if self.handler.free:
                    self.handler.free = False
                else:
                    processed_amount += 1
                    queue_length -= 1
                    return_chance = random()
                    if return_chance <= self.return_chance:
                        queue_length += 1

                if handler_was_free:
                    handling_time = previous_generated_time + self.handler.get_time_interval()
                else:
                    handling_time += self.handler.get_time_interval()
            else:
                self.handler.free = True
        current_time += self.step
    return max_length
\end{lstlisting}
\end{center}

\begin{center}
\captionsetup{justification=raggedright,singlelinecheck=off}
\begin{lstlisting}[label=lst:event-driven,caption=Реализация управляющей программы событийного принципа]
def event_driven(self):
        max_length = 0
        queue_length = 0
        processed_amounts = 0
        processed = False
        self.handler.free = True

        events = [[self.generator.get_time_interval(), generation]]

        while processed_amounts < self.messages_amount:
            event = events.pop(0)

            if event[state] == generation:
                queue_length += 1
                if queue_length > max_length:
                    max_length = queue_length
                self.__insert_event(events, [event[time] + self.generator.get_time_interval(), generation])
                if self.handler.free:
                    processed = True

            if event[state] == handling:
                processed_amounts += 1
                return_chance = random()
                if return_chance <= self.return_chance:
                    queue_length += 1
                processed = True
            if processed:
                if queue_length > 0:
                    queue_length -= 1
                    self.__insert_event(events, [event[time] + self.handler.get_time_interval(), handling])
                    self.handler.free = False
                else:
                    self.handler.free = True
                processed = False
        return max_length
\end{lstlisting}
\end{center}

\clearpage

\begin{center}
\captionsetup{justification=raggedright,singlelinecheck=off}
\begin{lstlisting}[label=lst:normal,caption=Вычисление интервала времени между появлениями сообщений по равномерному закону распределения]
def get_time_interval(self):
        return self.a + (self.b - self.a) * random()
\end{lstlisting}
\end{center}

\begin{center}
\captionsetup{justification=raggedright,singlelinecheck=off}
\begin{lstlisting}[label=lst:erlang,caption=Вычисление времени обработки сообщения по закону распределения Эрланга]
def get_time_interval(self):
        random_sum = sum([log(1 - random()) for _ in range(self.k)])
        chance =  -1 / (self.k * self.lambd) * random_sum
        print(chance)
        return chance
\end{lstlisting}
\end{center}

\subsection*{Примеры работы}

На рисунках \ref{img:example-1} и \ref{img:example-2} представлены примеры работы разработанной программы для описанной СМО без возврата обработанных сообщений и с возвратом.

\imgHeight{100mm}{example-1}{Пример работы программы, введённые вручную значения равны единице}

\imgHeight{100mm}{example-2}{Пример работы программы, введённые вручную значения отличаются друг от друга на константу}

\section*{Вывод}

В ходе выполнения лабораторной работы была реализована программа с графическим интерфейсом для моделирования системы массового обслуживания (СМО) при помощи принципа $\Delta t$ и событийного принципа и определения максимальной длины очереди, при которой не будет потери сообщений. 

