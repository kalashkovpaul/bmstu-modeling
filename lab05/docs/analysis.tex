
Целью данной работы является разработка программы с графическим интерфейсом для моделирования процесса обработки 300 запросов клиентов информационным центром и определения вероятности отказа клиенту в обслуживании. Информационный центр работает следующим образом:

\begin{enumerate}
	\item Клиенты приходят через интервал времени, равный 10 $\pm$ 2 мин.
	\item Если все три имеющихся оператора заняты, клиенту отказывают в обслуживании. Операторы имеют разную производительность и могут обеспечивать обслуживание среднего запроса пользователя за 20 $\pm$ 5 мин., 40 $\pm$ 10 мин. и 40 $\pm$ 20 мин. соответственно. Клиенты стремятся занять свободного оператора с максимальной производительностью.
	\item Полученные запросы сдаются в приемный накопитель, из которого они выбираются для обработки. На первый компьютер выбираются запросы от первого и второго операторов, на второй компьютер --- от третьего оператора. Время обработки на первом и втором компьютерах равны соответственно 15 мин. и 30 мин.
\end{enumerate}

В процессе взаимодействия клиентов и информационного центра предусмотреть: режим нормального обслуживания, когда клиент выбирает одного из свободных операторов c максимальной производительностью, и режим отказа.
 
\section*{Моделирование функционирования системы}

При моделировании функционирования системы эндогенными переменными являются:

\begin{itemize}
	\item время обслуживания клиента $i$-ым оператором, где $i = \overline{1, 3}$;	
	\item время обработки запроса на $j$-ом компьютере, где $j = \overline{1, 2}$.
\end{itemize}

\clearpage

Экзогенными переменными являются:

\begin{itemize}
	\item число обслуженных клиентов $n_{0}$;
	\item число клиентов, получивших отказ, $n_{1}$.
\end{itemize}

Уравнение модели имеет следующий вид:

\begin{equation}
    P_{\text{отказа}} = \frac{n_{1}}{n_{0} + n_{1}}
\end{equation}

\section*{Cхемы модели}

На рисунке \ref{img:struct} показана структурная схема модели.

\imgHeight{50mm}{struct}{Структурная схема модели информационного центра}

На рисунке \ref{img:system} представлена схема модели в терминах СМО.

\imgHeight{60mm}{system}{Схема модели в терминах СМО}

\section*{Результаты работы}

\subsection*{Детали реализации}

В листинге \ref{lst:client-generator} представлена реализация работы генератора клиентов, а в листингах \ref{lst:operator-1}--\ref{lst:operator-2} --- реализация работы оператора.
Реализации работы компьютера и самого информационного центра представлены в листингах \ref{lst:computer} и \ref{lst:center-1}--\ref{lst:center-2} соотвественно.
\clearpage

\begin{center}
\captionsetup{justification=raggedright,singlelinecheck=off}
\begin{lstlisting}[label=lst:client-generator,caption=Реализация работы генератора клиентов]
class ClientGenerator:
    def __init__(self, time_value, time_limit, operators, number):
        self.generator = TimeGenerator(time_value - time_limit, time_value + time_limit)
        self.operators = sorted(operators, key= lambda operator: operator.max_time)
        self.time_next = 0
        self.number = number

    def generate_client(self, time_prev):
        self.time_next = time_prev + self.generator.get_interval()

    def choose_operator(self):
        for operator in self.operators:
            if operator.is_free():
                return operator
        return None
\end{lstlisting}
\end{center}

\begin{center}
\captionsetup{justification=raggedright,singlelinecheck=off}
\begin{lstlisting}[label=lst:operator-1,caption=Реализация работы оператора (часть 1)]
class Operator:
    def __init__(self, time_value, time_limit, computer):
        self.time_generator = TimeGenerator(time_value - time_limit, time_value + time_limit)
        self.computer = computer
        self.max_time = time_value + time_limit
        self.time_next = 0
        self.free = True

    def generate_time(self, prev_time):
        self.time_next = prev_time + self.time_generator.get_interval()

    def is_free(self):
        return self.free

    def set_free(self):
        self.free = True
\end{lstlisting}
\end{center}

\begin{center}
\captionsetup{justification=raggedright,singlelinecheck=off}
\begin{lstlisting}[label=lst:operator-2,caption=Реализация работы оператора (часть 2)]
    def set_busy(self):
        self.free = False

    def get_computer(self):
        return self.computer
\end{lstlisting}
\end{center}


\begin{center}
\captionsetup{justification=raggedright,singlelinecheck=off}
\begin{lstlisting}[label=lst:computer,caption=Реализация работы компьютера]
class Computer:
    def __init__(self, time_value, time_limit):
        self.time_generator = TimeGenerator(time_value - time_limit, time_value + time_limit)
        self.queue = []
        self.time_next = 0
        self.free = True

    def generate_time(self, prev_time):
        self.time_next = prev_time + self.time_generator.get_interval()

    def is_free(self):
        return self.free

    def set_free(self):
        self.free = True

    def set_busy(self):
        self.free = False

    def is_empty(self):
        if self.queue:
            return False
        return True

    def add_request(self):
        self.queue.append(request)

    def pop_request(self):
        self.queue.pop(0)
\end{lstlisting}
\end{center}

\begin{center}
\captionsetup{justification=raggedright,singlelinecheck=off}
\begin{lstlisting}[label=lst:center-1,caption=Реализация работы информационного центра (часть 1)]
class Center:
    def __init__(self, client_generator):
        self.client_generator = client_generator

    def service_clients(self):
        failures = 0
        self.client_generator.generate_client(0)
        generated_clients = 1
        events = [Event(self.client_generator, self.client_generator.time_next)]

        while generated_clients < self.client_generator.number:
            events = sort_events(events)
            event = events.pop(0)

            if isinstance(event.creator, ClientGenerator):
                operator = self.client_generator.choose_operator()
                if operator is None:
                    failures += 1
                else:
                    operator.set_busy()
                    operator.generate_time(event.time)
                    events.append(Event(operator, operator.time_next))
                self.client_generator.generate_client(event.time)
                generated_clients += 1
                events.append(Event(self.client_generator, self.client_generator.time_next))

            elif isinstance(event.creator, Operator):
                operator = event.creator
                operator.set_free()
                computer = operator.get_computer()
                computer.add_request()
                if computer.is_free() and not computer.is_empty():
                    computer.pop_request()
                    computer.set_busy()
                    computer.generate_time(event.time)
                    events.append(Event(computer, computer.time_next))
\end{lstlisting}
\end{center}

\begin{center}
\captionsetup{justification=raggedright,singlelinecheck=off}
\begin{lstlisting}[label=lst:center-2,caption=Реализация работы информационного центра (часть 2)]
            elif isinstance(event.creator, Computer):
                computer = event.creator
                computer.set_free()
                if not computer.is_empty():
                    computer.pop_request()
                    computer.set_busy()
                    computer.generate_time(event.time)
                    events.append(Event(computer, computer.time_next))

        return failures

\end{lstlisting}
\end{center}

\subsection*{Примеры работы}

На рисунке \ref{img:example}  представлен пример работы разработанной программы для описанного информационного центра с обработкой 300 заявок.

\imgHeight{100mm}{example}{Пример работы программы, обработка 300 заявок}

\section*{Вывод}

В ходе выполнения лабораторной работы была реализована программа с графическим интерфейсом для моделирования процесса обработки 300 запросов клиентов информационным центром и определения вероятности отказа клиенту в обслуживании. Вероятность отказа равна примерно  0.2.

