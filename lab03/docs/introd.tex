\chapter*{Введение}
\addcontentsline{toc}{chapter}{Введение}

В программировании, как и в математике, часто приходится прибегать к использованию матриц.
Существует огромное количество областей их применения в этих сферах.
Матрицы активно используются при выводе различных формул в физике, таких, как:
\begin{itemize}[label=---]
    \item градиент;
    \item дивергенция;
    \item ротор.
\end{itemize}

Нельзя обойти стороной и различные операции над матрицами -- сложение, возведение в степень, умножение.
При различных задачах размеры матрицы могут достигать больших значений, поэтому оптимизация операций работы над матрицами является важной задачей в программировании.
В данной лабораторной работе пойдёт речь об оптимизациях операции умножения матриц.


\textbf{Целью данной работы} является изучение, реализация и исследование алгоритмов умножения матриц -- классический алгоритм, алгоритм Винограда и оптимизированный алгоритм Винограда.
Для достижения поставленной цели необходимо выполнить следующие задачи:
\begin{enumerate}[label=\arabic*)]
	\item изучить и реализовать алгоритмы -- классический, Винограда и его оптимизацию;
    \item протестировать перечисленные алгоритмы по времени и по памяти;
    \item сравнить и проанализировать время работы реализаций классического алгоритма и алгоритма Винограда;
    \item сравнить и проанализировать время работы реализаций алгоритма Винограда и его оптимизации;
	\item описать и обосновать полученные результаты в отчёте о выполненной лабораторной работе, выполненном как расчётно-пояснительная записка к работе.
\end{enumerate}
