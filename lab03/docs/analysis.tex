 Целью данной работы является разработка программы с графическим интерфейсом для генерации последовательностей псевдослучайных чисел с использованием алгоритмического и табличного способов и определения коэффициента случайности полученных последовательностей по разработанному критерию случайности. Предусмотреть возможность ввода последовательности вручную.

\section*{Генерация последовательностей}

\subsection*{Алгоритмический способ}

В качестве алгоритмического способа генерации псевдослучайных чисел был выбран способ генерации при помощи генератора равномерных вихревых последовательностей целых случайных величин без запоминающего массива.

Данный способ описан Алексеем Фёдоровичем Деоном в статье ''Генератор равномерных вихревых последовательностей случайных величин без запоминающего массива'', а также в статье ''Вихревой генератор случайных величин Пуассона  по технологии кумулятивных частот'', изданных в журнале ''Вестник приборостроения'' в 2020 году в МГТУ им. Н. Э. Баумана.

	

\subsection*{Табличный способ}

Табличный способ подразумевает использование файла (таблицы), содержащего случайные числа.
В данной работе использовались числа, приведённые в книге ''A million random digits with 100,000 normal deviates'' от корпорации RAND, выпущенной в 1955 году. Обход описанной таблицы осуществлялся сверху вниз и слева направо.

\section*{Критерий случайности}

Был составлен следующий критерий случайности последовательности: вычислялась b-арная энтропия $H_b(S)$ последовательности$S$ по формуле \ref{for:entropy}, вычислялось среднее арифметическое $k$ модулей разности чисел, расположенных рядом, делёных на максимальную из данных разностей. Итоговый коэффициент, определяющий случайность, вычислялся по формуле \ref{for:koeff}.

\begin{equation}
\label{for:entropy}
 H_b(S) = - \sum_{i=1}^{n} p_i \log_b p_i
\end{equation}

где $n$ --- количество встречающихся в последовательности чисел, $p_i$ - частота появления $i$-го числа, $b$ --- длина последовательности.

\begin{equation}
\label{for:koeff}
 r = \frac{1 - H_b(s) + k}{2}
\end{equation}


Чем ближе к нулю находится значение коэффициента $r$, тем случайнее значения последовательности $S$.
Данный критерий позволяет отслеживать как частоту появления числа в последовательности, так и характер изменения значений.

\section*{Результаты работы}

\subsection*{Детали реализации}

В листинге \ref{lst:table} представлена реализация табличного способа получения псевдослучайных величин, в листинге \ref{lst:algorithm} реализация алгоритмического спосооба, основанного на генераторе равнмерных вихревых последовательностей. 
\clearpage

\begin{center}
\captionsetup{justification=raggedright,singlelinecheck=off}
\begin{lstlisting}[label=lst:table,caption=Реализация табличного способа]
def read_s(self, digits, total):
    s = []
    divider = pow(10, digits)
    while total:
        item = self.table.read(6)
        if item == '':
            self.table.seek(0)
            item = self.table.read(6)
        item = int(item[:5])
        while item:
            if total:
                random = item % divider

                if len(str(random)) == digits:
                    s.append(random)
                    total -= 1

                item //= 10
            else:
                return s

    return s
\end{lstlisting}
\end{center}

\begin{center}
\captionsetup{justification=raggedright,singlelinecheck=off}
\begin{lstlisting}[label=lst:algorithm,caption=Реализация алгоритимческого спосба]
def Next(self):
        flagNext = True
        while flagNext:
            if self.stG == 0:
                flagNext = self.__DeonYuli_Next0()
            elif self.stG == 1:
                flagNext = self.__DeonYuli_Next1()
            elif self.stG == 2:
                flagNext = self.__DeonYuli_Next2()
        return self.xG
\end{lstlisting}
\end{center}

\subsection*{Примеры работы}

На рисунках \ref{img:example-1} и \ref{img:example-2} представлены примеры работы разработанной программы для систем из трёх и пяти состояний.

\imgHeight{70mm}{example-1}{Пример работы программы, введённые вручную значения равны единице}

\imgHeight{70mm}{example-2}{Пример работы программы, введённые вручную значения отличаются друг от друга на константу}

\section*{Вывод}

В ходе выполнения лабораторной работы была реализована программа с графическим интерфейсом для генерации последовательностей псевдослучайных чисел с использованием алгоритмического и табличного способов и определения коэффициента случайности полученных последовательностей по разработанному критерию случайности.

