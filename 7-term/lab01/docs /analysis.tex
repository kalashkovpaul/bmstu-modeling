 Целью данной работы является реализация программы для построения графиков функций и плотностей равномерного распределения и распределения Эрланга (вариант 3)для заданных значений параметров. 

\section*{Равномерное распределение}
	
Равномерное распределение --- распределение случайной величины, принимающей значения, принадлежащие некоторому промежутку конечной длины, характеризующееся тем, что плотность вероятности на этом промежутке всюду постоянна.

Функция распределения вероятности для равномерного распределения:

\begin{equation*}
F_X (x) =
    \begin{cases}
        0, x < a \\
        \frac{x - a}{b - a}, a \le x < b \\
        1, x \geq b \\
    \end{cases}
\end{equation*}
	
Функция плотности распределения вероятности для равномерного распределения:

\begin{equation*}
    f_X (x) =
    \begin{cases}
        \frac{1}{b-a}, x \in [a,b] \\
        0, x \notin [a, b] \\
    \end{cases}
\end{equation*}

\imgHeight{55mm}{equal}{Графики функций распределения и плотности распределения в общем виде}


Разработанная программа позволяет простроить графики функций распределения и плотности равномерного распределения для конкретных значений параметров $a$ и $b$:

\imgHeight{70mm}{equal_functions}{Графики функций распределения и плотности распределения для равномерного распредления и конкретных значений параметров}

\section*{Распределение Эрланга}

Распределение Эрланга --- это гамма-распределение  с параметром $k$, принимающим лишь целые положительные значения. 

Функция распределения вероятности для распределения Эрланга:

\begin{equation*}
F_X(x) = 1 - \sum_{i=0}^k  \frac{1}{i!} e^{-\lambda x} (\lambda x)^n
\end{equation*}
	
Функция плотности распределения вероятности для распределения Эрланга:

\begin{equation*}
f_X(x) = \frac{\lambda^k x^{k-1} e^{-\lambda x} } {(k-1)!}
\end{equation*}

В распределении Эрланга целочисленный положительный параметр $k$ 	--- параметр формы (т. е. он влияет на форму распределения,а не просто сдвигает его, как параметр местоположения, или растягивает его или сжимает, как параметр масштаба), а параметр $\lambda$ --- параметр скорости (т. е. он обратен параметру масштаба, отвечающему за растягивание или сжатие графика распределения). 

Графики функций распределения вероятности и плотности вероятости распределения Эрланга выглядят следующим образом:

\imgHeight{70mm}{erlang-distribution}{График функции распределения вероятности для распределения Эрланга}

\imgHeight{70mm}{erlang-density}{График функции распределения вероятности для распределения Эрланга}

Разработанная программа позволяет простроить графики функций распределения и плотности равномерного распределения для конкретных значений параметров $k$ и $\lambda$:

\imgHeight{70mm}{erlang_functions}{Графики функций распределения и плотности распределения для распределения Эрланга и конкретных значений параметров}

\section*{Вывод}

В ходе выполнения лабораторной работы была реализована программма для построения графиков функций и плотностей равномерного распределения и распределения Эрланга для заданных значений параметров.
Были построены и приведены графики при различных значениях параметров $a$, $b$ для равномерного распределения и параметров $\lambda$ и $k$ для распределения Эрланга.
