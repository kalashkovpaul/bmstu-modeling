
Целью данной работы является разработка программы на языке моделирования GPSS для моделирования процесса обработки 300 запросов клиентов информационным центром и определения вероятности отказа клиенту в обслуживании. Информационный центр работает следующим образом:

\begin{enumerate}
	\item Клиенты приходят через интервал времени, равный 10 $\pm$ 2 мин.
	\item Если все три имеющихся оператора заняты, клиенту отказывают в обслуживании. Операторы имеют разную производительность и могут обеспечивать обслуживание среднего запроса пользователя за 20 $\pm$ 5 мин., 40 $\pm$ 10 мин. и 40 $\pm$ 20 мин. соответственно. Клиенты стремятся занять свободного оператора с максимальной производительностью.
	\item Полученные запросы сдаются в приемный накопитель, из которого они выбираются для обработки. На первый компьютер выбираются запросы от первого и второго операторов, на второй компьютер --- от третьего оператора. Время обработки на первом и втором компьютерах равны соответственно 15 мин. и 30 мин.
\end{enumerate}

В процессе взаимодействия клиентов и информационного центра предусмотреть: режим нормального обслуживания, когда клиент выбирает одного из свободных операторов c максимальной производительностью, и режим отказа.
 
\section*{Моделирование функционирования системы}

При моделировании функционирования системы эндогенными переменными являются:

\begin{itemize}
	\item время обслуживания клиента $i$-ым оператором, где $i = \overline{1, 3}$;	
	\item время обработки запроса на $j$-ом компьютере, где $j = \overline{1, 2}$.
\end{itemize}

\clearpage

Экзогенными переменными являются:

\begin{itemize}
	\item число обслуженных клиентов $n_{0}$;
	\item число клиентов, получивших отказ, $n_{1}$.
\end{itemize}

Уравнение модели имеет следующий вид:

\begin{equation}
    P_{\text{отказа}} = \frac{n_{1}}{n_{0} + n_{1}}
\end{equation}

\section*{Cхемы модели}

На рисунке \ref{img:struct} показана структурная схема модели.

\imgHeight{50mm}{struct}{Структурная схема модели информационного центра}

На рисунке \ref{img:system} представлена схема модели в терминах СМО.

\imgHeight{60mm}{system}{Схема модели в терминах СМО}

\section*{Результаты работы}

\subsection*{Детали реализации}

В листингах \ref{lst:gpss-1}--\ref{lst:gpss-2} приведена реализация программы моделирования информационного центра на языке GPSS.
\clearpage

\begin{center}
\captionsetup{justification=raggedright,singlelinecheck=off}
\begin{lstlisting}[label=lst:gpss-1,caption=Реализация программмы на языке GPSS, часть 1]
client_medium FVARIABLE 10
delta_client FVARIABLE 2
operator_1_medium FVARIABLE 20
operator_1_delta FVARIABLE 5
operator_2_medium FVARIABLE 40
operator_2_delta FVARIABLE 10
operator_3_medium FVARIABLE 40
operator_3_delta FVARIABLE 20

computer_1_medium FVARIABLE 8
computer_1_delta FVARIABLE 7
computer_2_medium FVARIABLE 16
computer_2_delta FVARIABLE 14

GENERATE (v$client_medium),(v$delta_client)

TEST E f$operator_1,0,test_operator_2
SPLIT 1,processed
SEIZE operator_1
ADVANCE (v$operator_1_medium),(v$operator_1_delta)
RELEASE operator_1
TRANSFER ,to_queue_of_computer_1

test_operator_2 TEST E f$operator_2,0,test_operator_3
SPLIT 1,processed
SEIZE operator_2
ADVANCE (v$operator_2_medium),(v$operator_2_delta)
RELEASE operator_2
TRANSFER ,to_queue_of_computer_1

test_operator_3 TEST E f$operator_3,0,denied
SPLIT 1,processed
SEIZE operator_3
ADVANCE (v$operator_3_medium),(v$operator_3_delta)
RELEASE operator_3
TRANSFER ,to_queue_of_computer_2
\end{lstlisting}
\end{center}

\clearpage

\begin{center}
\captionsetup{justification=raggedright,singlelinecheck=off}
\begin{lstlisting}[label=lst:gpss-2,caption=Реализация программмы на языке GPSS, часть 2]
to_queue_of_computer_1 QUEUE computer_1_queue
SEIZE computer_1
DEPART computer_1_queue
ADVANCE (v$computer_1_medium),(v$computer_1_delta)
RELEASE computer_1
TRANSFER ,delete

to_queue_of_computer_2 QUEUE computer_2_queue
SEIZE computer_2
DEPART computer_2_queue
ADVANCE (v$computer_2_medium),(v$computer_2_delta)
RELEASE computer_2
TRANSFER ,delete

processed TRANSFER ,calculate_result
calculate_result SAVEVALUE PERCENT,(n$denied/(n$denied+n$processed))
TERMINATE 1
denied TERMINATE 1
delete TERMINATE 0


START 300
\end{lstlisting}
\end{center}

\subsection*{Примеры работы}

На рисунках \ref{img:example-1}--\ref{img:example-2} представлен пример работы разработанной программы для описанного информационного центра с обработкой 300 заявок. Вычисленная вероятность отказа равна 0.227.

\clearpage

\imgHeight{180mm}{example-1}{Результат работы программы, часть 1}

\clearpage

\imgHeight{180mm}{example-2}{Результат работы программы, часть 2}

\section*{Вывод}

В ходе выполнения лабораторной работы была реализована программа  на языке моделирования GPSS для моделирования процесса обработки 300 запросов клиентов информационным центром и определения вероятности отказа клиенту в обслуживании. Вероятность отказа равна  0.227.

