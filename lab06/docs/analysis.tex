 Целью данной работы является разработка программы с графическим интерфейсом для моделирования системы массового обслуживания (СМО) при помощи  языка моделирования GPSS и определения максимальной длины очереди, при которой не будет потери сообщений. Рассматриваемая СМО состоит из генератора сообщений, очереди ожидающих обработки сообщений и обслуживающего аппарата (ОА). Генерация сообщений происходит по равномерному закону распределения, время обработки сообщений --- согласно закону распределения Эрланга. Необходимо предоставить возможности ручного задания необходимых параметров, а также возможности возврата обработанного сообщения в очередь обработки с заданной вероятностью.
 
\section*{Принципы работы управляющей программы модели}

\subsection*{Событийный принцип}

Характерным свойством, отличающим событийный принцип, является изменение состояний свойств в дискретные моменты времени, совпадающие с моментами поступления сигналов окончания или аварийных сигналов. 
Состояния всех блоков программной (имитационной) модели анализируются лишь в момент появления события.
Момент наступления нового события определяется минимальным значением из списка будущих событий, представляющих собой совокупность ближайшего изменения состояния каждого блока системы.

\section*{Используемые законы распределения}

\subsection*{Закон появления сообщений}

Согласно заданию лабораторной работы для генерации сообщений используется равномерный закон распределения.
Случайная величина имеет равномерное распределение на отрезке $[a, b]$, если её функция плотности $p(x)$ имеет вид:

\begin{equation}
	\label{for:equal-1}
    p(x) = 
    \begin{cases}
        \frac{1}{b - a}, \text{если } x \in [a, b],\\
        0, \text{иначе.} \\
    \end{cases}
\end{equation}

Функция распределения $F(x)$ равномерной случайной величины имеет вид:

\begin{equation}
	\label{for:equal-2}
    F(x) = 
    \begin{cases}
    	0, \text{если } x  < a, \\
        \frac{x - a}{b - a}, \text{если } a < x  b,\\
        1, \text{если } x > b. 
    \end{cases}
\end{equation}

Интервал времени между появлением $i$-ого и $(i - 1)$-ого сообщения по равномерному закону распределения вычисляется следующим образом:

\begin{equation}
	T_{i} = a + (b - a) \cdot R,
\end{equation}

\noindentгде $R$ --- псевдослучайное число от 0 до 1.

\section*{Закон обработки сообщений}

Для моделирования работы генератора сообщений в лабораторной работе используется распределение Эрланга. 
Случайная величина имеет распределение Эрланга, если её функция плотности $p(x)$ имеет вид:

\begin{equation}
p(x) = \frac{\lambda^k x^{k-1} e^{-\lambda x} } {(k-1)!}
\end{equation}

В распределении Эрланга целочисленный положительный параметр $k$ 	--- параметр формы (т. е. он влияет на форму распределения,а не просто сдвигает его, как параметр местоположения, или растягивает его или сжимает, как параметр масштаба), а параметр $\lambda$ --- параметр скорости (т. е. он обратен параметру масштаба, отвечающему за растягивание или сжатие графика распределения). 

Функция распределения $F(x)$ нормальной случайной величины имеет вид:

\begin{equation*}
F(x) = 1 - \sum_{i=0}^k  \frac{1}{i!} e^{-\lambda x} (\lambda x)^n
\end{equation*}

Интервал времени между появлением $i$-ого и $(i - 1)$-ого сообщения по равномерному закону распределения вычисляется следующим образом:

\begin{equation}
	T_{i} = - \frac{1}{k \lambda} \sum_{j = 1}^{k} ln (1 - R_i),
\end{equation}

\noindentгде $R_j$ --- псевдослучайное число от 0 до 1.

\section*{Результаты работы}

\subsection*{Детали реализации}

В листинге \ref{lst:gpss} представлена реализация программы на языке GPSS
\clearpage

\begin{center}
\captionsetup{justification=raggedright,singlelinecheck=off}
\begin{lstlisting}[label=lst:gpss,caption=Реализация программы на языке GPSS]
lambda FVARIABLE 5/100
k FVARIABLE 3
a FVARIABLE 1
b FVARIABLE 10
ret FVARIABLE 0
time FVARIABLE 1000

GENERATE ((v$a+v$b)/2),((v$b-v$a)/2)
entered_queue QUEUE que
SEIZE server
DEPART que
ADVANCE (GAMMA(1,0,v$lambda,v$k))
RELEASE server
TRANSFER (v$ret),ex,entered_queue
ex TERMINATE 0


GENERATE v$time
TERMINATE 1
START 1
\end{lstlisting}
\end{center}

\subsection*{Пример работы}

На рисунках \ref{img:example-1} и \ref{img:example-2} представлены примеры работы разработанной программы для описанной СМО без возврата обработанных сообщений и с возвратом.

\imgHeight{200mm}{example-1}{Пример работы программы, введённые вручную значения равны единице}

\imgHeight{200mm}{example-2}{Пример работы программы, введённые вручную значения отличаются друг от друга на константу}

\clearpage

\section*{Вывод}

В ходе выполнения лабораторной работы была реализована программа на языке моделирования GPSS для моделирования системы массового обслуживания (СМО) и определения максимальной длины очереди, при которой не будет потери сообщений. 

