\chapter{Технологическая часть}

В данном разделе будут рассмотрены средства реализации, а также представлены листинги реализаций алгоритмов умножения матриц - стандартного, Винограда и оптимизированного алгоритма Винограда.

\section{Средства реализации}
В данной работе для реализации был выбран язык программирования $Python$~\cite{python-lang}. В текущей лабораторной работе требуется замерить процессорное время работы выполняемой программы и визуализировать результаты при помощи графиков. Инструменты для этого присутствуют в выбранном языке программирования.

Время работы было замерено с помощью функции \textit{process\_time(...)} из библиотеки $time$~\cite{python-lang-time}.


\section{Реализации алгоритмов}

В листингах \ref{lst:stand_alg}-\ref{lst:opt_vin_alg} представлены реализации алгоритмов умножения матриц --- стандартного, Винограда и оптимизированного алгоритма Винограда.

\begin{center}
    \captionsetup{justification=raggedright,singlelinecheck=off}
    \begin{lstlisting}[label=lst:stand_alg,caption=Реализация стандартного алгоритма умножения матриц]
def stdAlg(matA: Matrix, matB: Matrix) -> Matrix:
	if (len(matA[0]) != len(matB)):
		return []
	n = len(matA)
	m = len(matA[0])
	t = len(matB[0])
	result = [[0] * t for _ in range(n)]
	for i in range(n):
		for j in range(m):
			for k in range(t):
				result[i][j] += matA[i][k] * matB[k][j]
	
	return result
\end{lstlisting}
\end{center}


\begin{center}
    \captionsetup{justification=raggedright,singlelinecheck=off}
    \begin{lstlisting}[label=lst:vin_alg,caption=Реализация алгоритма Винограда умножения матриц]
def winAlg(matA: Matrix, matB: Matrix) -> Matrix:
	if (len(matA[0]) != len(matB)):
		return []
	
	n = len(matA)
	m = len(matA[0])
	t = len(matB[0])
	
	result = [[0] * t for _ in range(n)]
	
	rowSum = [0] * n
	colSum = [0] * t
	
	for i in range(n):
		for j in range(0, m // 2):
			rowSum[i] = rowSum[i] + matA[i][2 * j] * matA[i][2 * j + 1]
			
	for i in range(t):
		for j in range(0, m // 2):
			colSum[i] = colSum[i] + matB[2 * j][i] * matB[2 * j + 1][i]
	
	
	for i in range(n):
		for j in range(t):
			result[i][j] = -rowSum[i] - colSum[j]
			for k in range(0, m // 2):
				result[i][j] = result[i][j] + (matA[i][2 * k + 1] + matB[2 * k][j]) * (matA[i][2 * k] + matB[2 * k + 1][j])
	
	if (m % 2 == 1):
		for i in range(n):
			for j in range(t):
				result[i][j] = result[i][j] + matA[i][m - 1] * matB[m - 1][j]
	
	return result
\end{lstlisting}
\end{center}


\begin{center}
    \captionsetup{justification=raggedright,singlelinecheck=off}
    \begin{lstlisting}[label=lst:opt_vin_alg,caption=Реализация оптимизированного алгоритма Винограда умножения матриц]
def optWinAlg(matA: Matrix, matB: Matrix) -> Matrix:
	if (not len(matA) or len(matA[0]) != len(matB)):
		return []
	
	n = len(matA)
	m = len(matA[0])
	t = len(matB[0])
	
	result = [[0] * t for _ in range(n)]
	
	rowSum = [0] * n
	colSum = [0] * t
	
	halfM = m // 2
	
	for i in range(n):
		for j in range(0, halfM):
			rowSum[i] += matA[i][j << 1] * matA[i][(j << 1) + 1]
			
	for i in range(t):
		for j in range(0, halfM):
			colSum[i] += matB[j << 1][i] * matB[(j << 1) + 1][i]
	
	
	for i in range(n):
		for j in range(t):
			result[i][j] = -rowSum[i] - colSum[j]
			for k in range(0, halfM):
				result[i][j] += (matA[i][(k << 1) + 1] + matB[k << 1][j]) * (matA[i][k << 1] + matB[(k << 1) + 1][j])
	
	if (m % 2 == 1):
		for i in range(n):
			for j in range(t):
				result[i][j] += + matA[i][m - 1] * matB[m - 1][j]
	
	return result
\end{lstlisting}
\end{center}



\section{Функциональные тесты}

В таблице \ref{tbl:functional_test} приведена методология тестирования функций, реализующих алгоритмы умножения матриц, рассматриваемых в данной лабораторной работе. Тесты для всех алгоритмов пройдены \textit{успешно}.


\begin{table}[h]
	\begin{center}
		\begin{threeparttable}
		\captionsetup{justification=raggedright,singlelinecheck=off}
		\caption{\label{tbl:functional_test} Функциональные тесты}
		\begin{tabular}{|c@{\hspace{7mm}}|c@{\hspace{7mm}}|c@{\hspace{7mm}}|c@{\hspace{7mm}}|c@{\hspace{7mm}}|c@{\hspace{7mm}}|}
			\hline
			Матрица 1 & Матрица 2 & Ожидаемый результат \\ 
			\hline

			$\begin{pmatrix}
				1 & 5 & 7\\
				2 & 6 & 8\\
				3 & 7 & 9
			\end{pmatrix}$ &
			$\begin{pmatrix}
				&
			\end{pmatrix}$ &
			Сообщение об ошибке \\ \hline

			$\begin{pmatrix}
				1 & 2 & 3
			\end{pmatrix}$ &
			$\begin{pmatrix}
				1 & 2 & 3
			\end{pmatrix}$ &
			Сообщение об ошибке \\ \hline

			$\begin{pmatrix}
				1 \\
				1 \\
				1
			\end{pmatrix}$ &
		    $\begin{pmatrix}
		    	1 & 1 & 1
		    \end{pmatrix}$ &
			$\begin{pmatrix}
				1 & 1 & 1\\
				1 & 1 & 1 \\
				1 & 1 & 1
			\end{pmatrix}$ \\ \hline

			$\begin{pmatrix}
				1 & 2 & 3 \\
				4 & 5 & 6 \\
				7 & 8 & 9
			\end{pmatrix}$ &
			$\begin{pmatrix}
				1 & 0 & 0 \\
				0 & 1 & 0 \\
				0 & 0 & 1
			\end{pmatrix}$ &
			$\begin{pmatrix}
				1 & 2 & 3 \\
				4 & 5 & 6 \\
				7 & 8 & 9
			\end{pmatrix}$ \\ \hline

			$\begin{pmatrix}
				7
			\end{pmatrix}$ &
			$\begin{pmatrix}
				8
			\end{pmatrix}$ &
			$\begin{pmatrix}
				56
			\end{pmatrix}$ \\ \hline

		\end{tabular}
		\end{threeparttable}
	\end{center}
	
\end{table}

\section*{Вывод}

Были представлены листинги реализаций всех алгоритмов умножения матриц -- стандартного, Винограда и оптимизированного алгоритма Винограда. Также в данном разделе была приведена информации о выбранных средствах для разработки алгоритмов.
